\textbf{Une brève introduction aux concepts.}\newline

L’enjeux principal que nous avons rencontré lors de notre stage à été celui de la normalisation des données pour répondre aux besoins de l’agrégation.  \newline

Comme décrit plus tôt, la normalisation, en général, est un processus qui consiste à établir des critères ou des standards pour rendre les données homogènes, interopérables et facilement exploitables. En effet, se baser sur des critères communs permet ensuite à l’utilisateur de faire des comparaisons entre les données, de savoir les rechercher, et donc de pouvoir accéder à l’information. Dans le contexte des collections culturelles, ce processus revêt une importance cruciale, car il permet de structurer des données souvent disparates, collectées sous diverses formes, et de les rendre accessibles à travers des systèmes d’information interconnectés. \newline

Il faut noter que les données disparates sont le reflet des collections qu’elles représentent. Les collections culturelles, notamment celles conservées dans les musées peuvent être extrêmement variées, allant d’objets de beaux arts plus “classiques”, à des objets plus inhabituels, au sein de collections ethnographiques par exemple. 
Au-delà des objets divers qui peuvent être conservés, et donc mis en données, les données culturelles concernent aussi la vie de l’objet dans sa création: des dates, des titres (parfois plusieurs), des noms, des lieux etc… mais aussi la vie de l’objet dans les collections, avec des numéros d’inventaire, un historique d'appartenance, des données relevant de ses conditions de conservation, etc… \newline

Avant de parler plus en profondeur de la normalisation nous allons préciser au mieux la notion de “donnée”. En effet, celle-ci est au centre de toutes nos réflexions, mais il nous est difficile de lui trouver une définition qui représente toute sa complexité.
Dans son ouvrage \textit{Qu’est-ce que le travail scientifique des données ?}, Christine L. Borgman, convient elle-même que \begin{quote}
    <<Malgré ses cinq siècles d’existence, le terme data ou « donnée » n’a toujours pas trouvé une définition consensuelle.” Mais elle ajoute que “La synthèse la plus exhaustive consiste à dire que les données sont des représentations d’observations, d’objets ou d’autres entités qui servent à mettre en évidence des phénomènes à des fins de recherche.>> \footcite{borman_2020}
\end{quote}
La donnée est donc la représentation d’une certaine réalité à partir de laquelle on peut extraire un sens et des connaissances.
