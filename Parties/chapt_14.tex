\subsection{Pourquoi avoir choisi LIDO ?}

Tout d'abord, comme nous l'avons expliqué dans notre introdution, le moment était propice à la mise en place de la stratégie d'agrégation. La création d'un service du numérique au Ministère de la Culture, à permis d'établir une infrastructure commune et a donc mid en exergue le besoin d'uniformiser les flux de données pour industrialiser et massifier les quantités de données traitées par le ministère. Pour répondre a ce besoin c'est le LIDO qui est ressorti comme le modèle à utiliser.\newline

Il faut noter qu'il existait ben au ministère des formats d'échange de données culturelles: Le format d'export Joconde, le DC en OAI-PMH pour le moteur collection. mais ceux-ci étaient limités en termes de données, peu compatibles et ne proposaient aucun alignements vers d'autres standards.\newline

Le choix de LIDO comme standard de circulation pour l’agrégation des données muséales s’explique par plusieurs facteurs. Tout d’abord, LIDO est un modèle qui offre une interopérabilité élevée, ce qui est essentiel pour les musées qui souhaitent diffuser leurs collections à une échelle nationale ou internationale. En adoptant un modèle standardisé, les musées peuvent s’assurer que leurs données seront compatibles avec celles d’autres institutions, ce qui facilite l’agrégation et la diffusion des informations. \newline

Un autre avantage majeur de LIDO est sa simplicité. Contrairement à des modèles plus complexes comme CIDOC CRM, LIDO est basé sur le XML. Ce format est déjà utilisé dans de nombreux cas de figures, c'est le cas par exemple pour l'EAD en archives. Ce format est connu des archivistes et des documentalistes, ce qui pourrait le rendre plus ateignable pour des professionels des musées (même si c'est ne sont pas eux qui vont saisir les données en LIDO, ils pourront avoir une idée de la forme que l'export va prendre).\newline

LIDO est également un modèle conçu pour répondre aux besoins spécifiques des musées. Contrairement à d’autres standards plus génériques, LIDO prend en compte les spécificités des objets muséaux, en incluant des éléments comme les informations de provenance, les données contextuelles et les liens avec d’autres objets. Cela permet de capturer la complexité des collections muséales, tout en garantissant que ces informations peuvent être diffusées de manière standardisée. \newline

Un autre facteur qui a contribué au succès de LIDO est la gouvernance du modèle. Développé par l’ICOM et d’autres institutions culturelles internationales, LIDO bénéficie d’un soutien institutionnel fort, ce qui garantit la pérennité du modèle et sa capacité à évoluer pour répondre aux nouveaux défis du secteur culturel. En outre, LIDO est soutenu par une infrastructure technique solide, avec des outils et des ressources disponibles pour faciliter son adoption par les musées et les développeurs de systèmes de gestion de collections.
