\subsection{Introduction aux Modèles de Données : MARC et EAD}

Les modèles de données sont des schémas standardisés utilisés pour structurer et organiser les informations dans le domaine du patrimoine culturel. Parmi les plus courants, on trouve le format MARC (Machine-Readable Cataloging) et le format EAD (Encoded Archival Description). \newline

Le format MARC, créé dans les années 1960, est un modèle de données utilisé principalement dans les bibliothèques pour la description des ressources bibliographiques. Ce format permet de structurer les informations de manière à ce qu'elles soient lisibles par des machines, facilitant ainsi l'échange de données entre institutions. En revanche, le format EAD, apparu dans les années 1990, est destiné aux archives. Il permet de structurer la description des collections archivistiques de manière hiérarchique, rendant ainsi possible la recherche et la navigation au sein de ces collections.\newline

L’un des premiers modèles adoptés dans les bibliothèques et les archives est le MARC (Machine-Readable Cataloging), un format de description normalisé qui a été développé dans les années 1960 par la Bibliothèque du Congrès aux États-Unis. Ce modèle permet de décrire de manière uniforme les documents bibliographiques et a été largement adopté par les bibliothèques à travers le monde. Il a constitué la base pour le développement d’autres modèles plus récents, comme le Dublin Core, qui a été conçu pour être un modèle plus léger et plus adapté aux besoins du web sémantique.
Le Dublin Core est un schéma de métadonnées simple qui permet de décrire les objets numériques à travers 15 éléments de base, comme le titre, l’auteur, la date de création, le sujet, et la description. Ce modèle a été adopté dans de nombreux contextes, notamment pour la description des objets culturels dans des environnements numériques. Cependant, il reste limité en termes de granularité et de capacité à décrire des objets complexes, ce qui a conduit à la création de modèles plus spécialisés pour les musées, comme CIDOC CRM et LIDO.\newline

Ces modèles de données sont essentiels car ils permettent de passer de la donnée brute, souvent complexe et difficile à exploiter, à une information structurée et normalisée. Ce processus est crucial pour la diffusion et l’accessibilité des données sur le web, où l'interopérabilité des systèmes est une exigence fondamentale. \newline

\textbf{Le Lien entre Donnée Brute et Diffusion : L'Importance des Modèles Structurés}
Dans un contexte numérique, la donnée brute doit être transformée pour être exploitable dans des systèmes de diffusion. Cette transformation repose sur l’application de modèles de données comme MARC et EAD. Par exemple, une notice bibliographique en MARC peut être facilement partagée entre bibliothèques, tandis qu'une description archivistique en EAD peut être intégrée dans un portail d'archives en ligne. Cette interopérabilité est rendue possible grâce à la normalisation des données.\newline

En effet, un modèle de données permet de définir les champs et les valeurs possibles pour chaque type d'information, assurant ainsi que chaque donnée est conforme à un standard commun. Cela facilite non seulement la diffusion, mais aussi la recherche et la réutilisation des données à travers différentes plateformes. Les utilisateurs peuvent ainsi accéder à des informations provenant de multiples institutions, sans se soucier des différences de format ou de structure. \newline

Exemples Concrets d'Utilisation : MARC, EAD et leur Impact
L'utilisation de MARC dans les bibliothèques et d'EAD dans les archives a permis une révolution dans la manière dont les informations sont partagées et diffusées. Par exemple, la Bibliothèque nationale de France (BnF) utilise le format MARC pour cataloguer et diffuser ses collections via le portail Gallica. De même, les Archives nationales de France utilisent le format EAD pour structurer et publier leurs inventaires en ligne, rendant ainsi les documents archivistiques plus accessibles au public et aux chercheurs.\newline
Les modèles de données comme MARC et EAD sont des outils indispensables pour structurer et diffuser les informations culturelles sur le web. Ils permettent de transformer des données brutes en informations structurées, facilitant ainsi leur partage, leur recherche et leur réutilisation.\newline

Ces deux modèles que nous avont présenté, sont majoritairement utilisés par les bibliothèques et les services d'archives, nous allons maintenant nous intérresser aux musées.

