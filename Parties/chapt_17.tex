\subsection{Les propositions qui sont faites ne sont pas toujours satisfaisantes.}


Comme évoqué précédemment, nous avons rencontré des difficultés lors du mapping d'un modèle vers le modèle. C'est en ayant conscience de ces difficultés que nous devons expliquer les limites du travail que nous avons proposé. \newline

Si, pour la plupart des modèles que nous avons analysés, nous avons pu trouver des correspondances satisfaisantes entre leurs données et la structure imposée par le modèle LIDO, ce n'a pas été le cas pour tous. \newline

Les données aux formats Dublin Core ne semblaient pas poser problème à première vue car elles n'étaient structurées que suivant les 15 éléments du modèle (parfois moins). Mais nous avons rapidement réalisé ces 15 éléments qui sont, à dessein, conçus pour être vagues afin de pouvoir porter tous les types de données utilisées par l'institution, n'étaient pas toujours utilisés de la même manière en fonction des institutions. \newline

Cette situation a donc rendu impossible la réalisation d'un mappage Dublin Core-LIDO, général. En effet, pour éviter de perdre la qualité des données fournies par toutes les institutions (pour les raisons évoquées plus tôt) il aurait fallu faire un mappage adapté à l'usage de Dublin Core que font chaque institution. Autant dire faire un mappage par institutions car elles en font toutes un usage légèrement différent… \newline

Faute de temps, nous n'avons pas été en mesure de fournir ce travail. Le mapping proposé pour les données en Dublin Core est donc obligatoirement insatisfaisant pour tous. Nous avons essayé de fournir un document qui soit général et représente les différents cas de figure et comment les faire correspondre dans le modèle LIDO, mais celui à forcément ses limites et ne pourra servir qu'à titre d'exemple, pas en temps qu’outil pour convertir les données.\newline

Pour revenir sur les difficultés rencontrées pour le mappage du modèle RDF utilisé pour les données du projet CapData Opéra vers le modèle LIDO, nous n'avons finalement pas proposé de mapping. Après plusieurs échanges avec Eudes Peyre et des tentatives de modélisation pour trouver des correspondances, nous nous sommes rendus compte que la tâche était trop complexe et que nous manquons de temps. Nous nous sommes donc accordés sur le fait que, pour s'assurer de conserver la qualité des données et toute leur complexité, il faudrait revenir sur ce chantier plus tard, peut-être une fois que la première version du modèle LIDO ministère de la Culture serait publiée. \newline

Nous avons également conscience que les mapping proposé pour les données de l’export Joconde n’est probablement pas parfait. Même si les correspondances entre les deux modèles ont pu être trouvées sans problème, d'autres ont demandé plus d'efforts et d'échanges avec nos interlocutrices au SMF. Il faut retenir que les propositions faites sont les notres, mais qu'une autre personnes pourraient en faire d'autres.\newline

Il en va de même pour la documentation. Notre approche a été celle e produire un document unique pour pouvoir centralsier l'information et donc pouvoir faciliter et tracer les mises jour. Cette documentation est conçue pour s'adresser à un public le plus large possible, mais une approche différente n'est pas forcément erronée.\newline

\textbf{les limites du modèle LIDO}. \newline

Le modèle LIDO est un modèle pour decrire les objets
musées, donc qui n’est pas toujours idéal pour décrire des objects différents, des évements et autres type de notices. Hors, durant notre stage, ce sont des cas de figure que nous avons rencontré, notament lors de nos échanges avec la ROF.\newline

Pour répondre aux besoins de l'agrégation le modèle de circulation LIDO sera adapté à la pluparts des données collectées, mais nous avons conscience que les métiers utilisent parfais des modèles plus adaptés à leur données qui sont difficilement convertible en LIDO.\newline

Pour répondre à un maximum de besoins et collecter des données en masse, il faudra des infrastructure capables de prendre d'autres formats
