
\newglossaryentry{biblissima}{
  name={Biblissima},
  description={Un observatoire du patrimoine écrit, un projet de recherche français visant à valoriser les manuscrits et livres anciens.}
}

\newglossaryentry{bretania}{
  name={Bretania},
  description={Un portail qui agrège et diffuse des données d'institutions culturelles de Bretagne.}
}

\newglossaryentry{capdata_opera}{
  name={CapData Opéra},
  description={Un projet visant à faciliter l'interopérabilité des données des maisons d'opéra grâce aux technologies du web sémantique.}
}

\newglossaryentry{cidoc_crm}{
  name={CIDOC CRM},
  description={Conceptual Reference Model, une ontologie pour les informations patrimoniales et muséales.}
}

\newglossaryentry{crm}{
  name={CRM},
  description={Conceptual Reference Model, un modèle standard pour la documentation des objets culturels.}
}

\newglossaryentry{crm_dig}{
  name={CRMdig},
  description={Une extension du CIDOC CRM dédiée à la documentation des processus de numérisation et des objets numériques.}
}

\newglossaryentry{dc}{
  name={Dublin Core (DC)},
  description={Un standard pour la description des ressources numériques, particulièrement utilisé dans les bibliothèques et musées.}
}

\newglossaryentry{dublin_core}{
  name={Dublin Core},
  description={Un schéma de métadonnées utilisé pour la description d'objets numériques, largement adopté dans les bibliothèques et archives.}
}

\newglossaryentry{drac}{
  name={DRAC},
  description={Direction régionale des affaires culturelles, une administration publique en France chargée de la mise en œuvre de la politique culturelle du Ministère de la Culture.}
}

\newglossaryentry{ead}{
  name={EAD},
  description={Encoded Archival Description, un standard XML pour la description des documents d'archives.}
}

\newglossaryentry{europeana}{
  name={Europeana},
  description={Une plateforme en ligne fournissant un accès à des millions d'objets numériques provenant d'institutions culturelles européennes.}
}

\newglossaryentry{frbr}{
  name={FRBR},
  description={Functional Requirements for Bibliographic Records, un modèle conceptuel pour structurer les catalogues de bibliothèques.}
}


\newglossaryentry{isni}{
  name={ISNI},
  description={International Standard Name Identifier, un identifiant unique pour les créateurs d’œuvres.}
}

\newglossaryentry{iso}{
  name={ISO},
  description={Organisation internationale de normalisation, créatrice des standards utilisés dans de nombreux domaines, y compris la gestion des données culturelles.}
}

\newglossaryentry{iso21127}{
  name={ISO 21127},
  description={La norme ISO qui définit le CIDOC CRM pour la documentation des objets culturels.}
}

\newglossaryentry{joconde}{
  name={Joconde},
  description={La base de données nationale des collections des musées de France.}
}

\newglossaryentry{lido}{
  name={LIDO},
  description={Lightweight Information Describing Objects, un standard XML pour l'échange d'informations sur les objets culturels.}
}

\newglossaryentry{lod}{
  name={LOD},
  description={Linked Open Data, un concept permettant de relier des données ouvertes sur le Web en utilisant des standards comme RDF.}
}

\newglossaryentry{logilab}{
  name={Logilab},
  description={Une entreprise spécialisée dans le développement de logiciels libres pour la gestion des connaissances et des données.}
}

\newglossaryentry{opentheso}{
  name={Opentheso},
  description={Un logiciel open source pour la gestion de thésaurus multilingues.}
}

\newglossaryentry{owl}{
  name={OWL},
  description={Web Ontology Language, un langage standard du W3C pour représenter des connaissances riches et complexes.}
}

\newglossaryentry{pop}{
  name={POP},
  description={Plateforme Ouverte du Patrimoine, un portail permettant l'accès aux données culturelles françaises.}
}

\newglossaryentry{rdf}{
  name={RDF},
  description={Resource Description Framework, un modèle de données permettant la description de ressources sur le web.}
}

\newglossaryentry{schemaorg}{
  name={schema.org},
  description={Un vocabulaire standardisé pour structurer les données sur le web, largement utilisé pour l'optimisation SEO.}
}

\newglossaryentry{sicd}{
  name={SICD},
  description={Service Interétablissement de Coopération Documentaire, une structure de mutualisation des ressources documentaires entre plusieurs institutions.}
}

\newglossaryentry{skos}{
  name={SKOS},
  description={Simple Knowledge Organization System, un modèle RDF pour le partage de vocabulaires structurés.}
}

\newglossaryentry{sparql}{
  name={SPARQL},
  description={SPARQL Protocol and RDF Query Language, un langage de requête pour interroger les données RDF.}
}

\newglossaryentry{sparql_endpoint}{
  name={SPARQL Endpoint},
  description={Un point d'accès web où des requêtes SPARQL peuvent être exécutées pour interroger des bases de données RDF.}
}

\newglossaryentry{thesaurus}{
  name={Thésaurus},
  description={Un outil de vocabulaire contrôlé utilisé pour organiser et structurer des informations en muséographie et en bibliothéconomie.}
}

\newglossaryentry{w3c}{
  name={W3C},
  description={World Wide Web Consortium, une organisation qui développe des standards pour le web, incluant HTML, RDF, et OWL.}
}

\newglossaryentry{xml}{
  name={XML},
  description={Extensible Markup Language, un langage de balisage utilisé pour structurer des documents de manière hiérarchique.}
}
