\subsection{Les Thésaurus et Leur Contexte d'Utilisation
Définition et Importance des Thésaurus dans les Musées}

Les thésaurus sont des outils essentiels pour la normalisation des données dans les musées. Ils permettent de structurer et d'organiser les informations en utilisant des vocabulaires contrôlés, ce qui facilite la recherche, l'indexation et l'interopérabilité des données. Un thésaurus est généralement composé de termes hiérarchisés, liés à des relations synonymiques, antinomiques ou d'association, permettant ainsi de représenter de manière précise les concepts liés aux objets des collections.\newline

Dans le contexte muséal, les thésaurus sont utilisés pour décrire les objets, les techniques, les matériaux, les périodes historiques, et bien d'autres aspects des collections. Leur utilisation permet de garantir une homogénéité dans la description des objets, ce qui est crucial pour la recherche et l'agrégation des données à travers différentes institutions.\newline

\textbf{Situation Actuelle : Développements Internes et Normalisation Externe}
Actuellement, de nombreux musées développent leurs propres thésaurus en interne, adaptés aux spécificités de leurs collections. Cette approche présente l'avantage de répondre précisément aux besoins des institutions, mais elle pose également des défis en termes d'interopérabilité. En effet, ces thésaurus internes diffèrent presque systématiquement d'une institution à l'autre, ce qui complique grandement le partage et l'agrégation des données.\newline

Pour pallier ce problème, certaines institutions ont commencé à normaliser leurs thésaurus en les publiant en ligne et en les rendant accessibles à d'autres musées. Par exemple, l'ICOM et le Getty Research Institute ont développé des thésaurus largement utilisés et nommés respectivement  Getty Art \& Architecture Thesaurus (AAT)\footcite{aat-getty} et le Thesaurus of Geographic Names (TGN). 

Le Getty Thesaurus of Geographic Names (abrégé TGN) est un produit du J. Paul Getty Trust qui fait partie du Getty Vocabulary Program. Le TGN comprend des noms et des informations associées sur des lieux. Les lieux figurant dans le TGN comprennent des entités politiques administratives (des villes, des nations). Mais il comprends aussi des termes décrivant les caractéristiques physiques d'un lieu (des montagnes, des rivières). Les lieux actuels et historiques sont inclus (on pense à des entités qui n'existent plus, par exemple l'empire prussien, qui englobe des régions faisant partie de pays différents aujourd'hui). D'autres informations relatives à l'histoire, à la population, à la culture, à l'art et à l'architecture sont incluses.\footcite{thesaurus_getty_wiki}

Cette ressource est mise à la disposition des musées, des bibliothèques d'art, des archives, des catalogueurs de collections de ressources visuelles et des projets bibliographiques par le biais d'une licence privée. Elle est également disponible gratuitement pour le grand public sur le site Web du Vocabulaire Getty.

Les bases de données de vocabulaire du Getty (Art \& Architecture Thesaurus (AAT), Union List of Artist Names (ULAN), et TGN) sont produites et maintenues par le Getty Vocabulary Program. Elles sont conformes aux normes ISO et NISO pour la construction de thésaurus.

Ces thésaurus sont disponibles en ligne et permettent une normalisation à grande échelle, facilitant ainsi la collaboration entre institutions et l'échange de données.\footcite{getty_theso} \newline

\textbf{Exemples et Avantages de la Normalisation à Grande Échelle}\newline

L'utilisation de thésaurus normalisés à grande échelle présente plusieurs avantages. Par exemple, le Getty AAT est utilisé par de nombreux musées à travers le monde pour décrire les matériaux et les techniques des objets d'art. Cela permet une recherche plus efficace et une meilleure interopérabilité des données, car les termes utilisés sont les mêmes dans différentes bases de données.
Le Thesaurus of Geographic Names (TGN), également développé par le Getty, est utilisé pour normaliser les noms géographiques, ce qui est essentiel pour la description des lieux associés aux objets dans les collections muséales. En utilisant ces thésaurus, les musées peuvent garantir que les termes employés pour décrire des concepts, des techniques, ou des lieux sont uniformes à travers les institutions. Cela facilite grandement l'agrégation et la recherche des données, en particulier dans des environnements numériques où l'interopérabilité est cruciale.
L’intérêt de ces thésaurus normalisés réside non seulement dans l’homogénéité qu’ils apportent à la description des collections, mais aussi dans la facilité d’échange des informations entre les institutions. Par exemple, le fait que plusieurs musées utilisent le Getty AAT permet à des bases de données différentes d’interagir et de partager des informations sans confusion ni malentendu. Cela est particulièrement important dans le cadre de projets de numérisation massive où des objets provenant de diverses institutions doivent être comparés ou regroupés.
En outre, l'utilisation de thésaurus normalisés à grande échelle comme ceux du Getty permet également de faciliter l'intégration des données muséales dans des plateformes globales comme Europeana, qui rassemble des millions d'objets provenant de musées, bibliothèques et archives à travers l'Europe. L'adoption de vocabulaires communs rend non seulement cette agrégation possible mais en améliore également l'efficacité et la précision. \newline

Cependant, malgré ces avantages, l'utilisation de thésaurus normalisés n'est pas sans défis. L'un des principaux obstacles est l'adaptation de ces thésaurus aux spécificités locales ou institutionnelles. Par exemple, un musée spécialisé dans une culture ou une époque particulière pourrait avoir des besoins qui ne sont pas entièrement couverts par les thésaurus existants. Cela peut conduire à des divergences dans la manière dont les objets sont décrits, même lorsque des thésaurus normalisés sont utilisés. \newline

Pour surmonter ces défis, certaines institutions adoptent une approche hybride, combinant des thésaurus internes spécifiques avec des thésaurus normalisés externes. Une institution peut par exemple continuer à utiliser son thésaurus et quand même aligner les termes utilisés vers des termes provenant de thésaurus en ligne. Cette méthode permet de répondre aux besoins particuliers tout en maintenant une certaine compatibilité avec les standards internationaux. Des efforts sont également en cours pour enrichir les thésaurus existants en intégrant des termes et des concepts provenant de différentes cultures et disciplines, ce qui permet une plus grande inclusivité et pertinence.
