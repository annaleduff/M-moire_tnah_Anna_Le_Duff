\subsection{Pourquoi normaliser ? L’intérêt de la recherche et de l’interopérabilité}

La normalisation est indispensable pour garantir que les données, quelles que soient leurs formes, puissent être recherchées et récupérées de manière efficace. \newline

Pour les données culturelles, que ce soit dans le monde des bibliothèques, des archives, ou des musées, il y a l'exemple des noms d'auteurs ou des titres d'œuvres. Sans normalisation, un même auteur pourrait être référencé sous différentes orthographes ou variantes de noms, ces doublons rendent la recherche d'informations incohérente et fragmentée.  Il en va de même pour les noms de lieux, ou pour les dates qui peuvent être écrites de manières très variées (les systèmes jj/mm/aa et mm/jj/aa, ne sont qu’un petit exemple de la confusion que ça peut causer). En normalisant les noms d’auteurs, on assure que toutes les œuvres d’un même créateur soient regroupées sous une seule et même entrée, facilitant ainsi la recherche et l’analyse pour tous les chercheurs, conservateurs et autres utilisateurs qui en auraient besoin. \newline

De même, dans le cadre de l’interopérabilité, des données homogènes sont nécessaires pour que différents systèmes puissent « parler le même langage ». Par exemple, lorsqu'une institution veut partager ses collections avec une plateforme nationale ou internationale, il est essentiel que ses données soient structurées selon des standards communs. Cela garantit que les données puissent être intégrées sans ambiguïté dans un système plus vaste, permettant ainsi une meilleure exploitation et valorisation des informations. \newline

Le Catalogue Collectif de France (CCFr), réunit par exemple plusieurs bases de données, donc 3 de ses bases Manuscrit utilisent le format EAD \footcite{falconnet_sirdey_borda}(Format XML (DTD et schéma) utilisé pour l’encodage des descriptions de fonds d’archives, également utilisé en bibliothèque pour la description de manuscrits.)\newline

\textbf{La mesure : un autre domaine nécessitant une forte normalisation}\newline
Pour donner un exemple plus concret en dehors du domaine de la culture, on peut parler de la normalisation des données de mesure. Les dimensions, poids, ou volumes des objets doivent être enregistrés de manière standardisée pour permettre des comparaisons et des analyses précises. Si un système manufacture utilise les kilos et qu’on lui donne des données en livres, alors cela peut mener à des problèmes quand les produits ne correspondront pas aux attentes. Sans une normalisation rigoureuse, ces données seraient difficiles à interpréter et à comparer, limitant ainsi leur utilité dans un contexte industriel, scientifique  et même muséographique.

