\subsection{Historique de la normalisation, l’évolution des pratiques dans les bibliothèques et les archives.}

Avant d’aborder le sujet de la normalisation à grande échelle dans les musées, nous allons nous pencher sur l’historique de ce processus dans d'autres domaines culturels qui ont réfléchi à ces questions plus tôt . \newline

Comme nous l’avons évoqué précédemment, les bibliothèques et les archives ont été plus rapides dans l’adoption de ce processus. Au fil des décennies, bibliothèques, archives et musées ont accumulé d’énormes quantités de données, allant des descriptions physiques des objets aux informations contextuelles, telles que l’histoire des œuvres, leur créateur, leur signification culturelle, et les informations sur leur conservation. Ces instutions traitent souvent des fonds ou des documents faisant partie d'ensembles, pour s'y retrouver au sein de ceux-ci, les bibliothécaires on développé des outils. À l’origine, ces données étaient consignées sous forme de registres papier, catalogues ou inventaires. Ces systèmes fragmentés étaient destinés aux gestionnaires des collections, conservateurs et régisseurs par exemple. Ils étaient inaccessibles au grand public. \newline

La volonté de partage de l’information prend forme très tôt dans le domaine des bibliothèques. En effet, dès la Révolution française, on prend conscience q u'elles sont un axe central pour répondre au besoin d’instruction de la population. Le bibliothécaire du roi, Lefèvre d’Ormesson, fait approuver le projet d'établir un catalogue général des livres à partir des inventaires propres à chaque établissement, fin 1790. L’objectif est de constituer un catalogue collectif national : Bibliographie universelle de la France. \newline

En 1791, on donne des instructions pour réaliser les catalogues de chaque bibliothèque ecclésiastique. On peut considérer que ces instructions constituent en fait un premier code normalisé de catalogage. Pour les mettre en forme es références des ouvrages sont écrites sur les dos de cartes à jouer qui deveniennent alors des fiches de bibliothèques. Ainsi, l’état des fonds et des richesses enfouies sera connu afin de
\begin{quote}  
« […] rendre à la lumière, aux lettres et aux progrès de la raison humaine les monuments ensevelis ; les répartir avec justice entre les départements de l’Empire pour y être comme des phares de correspondance ; vendre, sans crainte d’erreur, les objets peu utiles ou multipliés, mais ne vendre que ceux-là ; donner à chaque dépôt sa bibliographie particulière et à l’Europe la bibliographie générale de la France, tel est en abrégé l’objet que le Comité s’étoit proposé ». \footcite{fayet_scribe_2000}
\end{quote}
Deux innovations dans les outils d'accès utilisés à la fin du 18ème siècle sont identifiées par Éric de Grolier, chercheur en linguistique et en scientométrie français. La première concerne les débats sur la classification bibliographique à l'Institut de France, où l'on discute de l'ordre naturel des connaissances en 1796. Selon lui, il s'agit probablement des premières discussions théoriques sur le système des sciences appliqué à la classification documentaire. La seconde concerne le domaine de la bibliographie nationale courante, en effet, le libraire parisien Pierre Roux fait paraître une bibliographie générale nationale du 22 septembre 1797 au 16 octobre 1810 en treize volumes. Cet ouvrage est le Journal typographique et bibliographique et c’est est l’ancêtre de l’actuelle Bibliographie de la France.\footcite{fayet_scribe_2000}\newline

C’est à la même période, dans la deuxième moitié du 18eme siècle, qu’on peut mentionner une autre innovation, le verbe « documenter » (1755) apparaît. Il possède le sens ancien correspondant à « to document » en anglais : instruire, enseigner. En français, le seul sens du mot jusqu’alors a été d’exprimer : ce qui sert à instruire, enseignement, leçon. Le sens moderne : écrit servant de preuve ou de renseignement, provient de l’emploi du mot comme terme juridique dans Titres et documents (1690). Les dérivés successifs au mot « document », « documenter » (en 1755), puis « documentation » (en 1870), puis encore « documentaire » (1877) et enfin « documentaliste » (1932), lui donnent le sens que nous lui connaissons aujourd'hui : tout écrit servant de preuve ou de renseignement. \newline

La plupart des moyens d'accès que nous connaissons aujourd'hui existent déjà  à la fin du XVIIIe siècle : catalogue, bibliographie, annuaire, encyclopédie, dictionnaire, etc. Ces derniers se multiplient et se développent à mesure que l'information scientifique et technique prend de l'ampleur grâce à la fois à l'évolution des différentes disciplines scientifiques et à l'industrialisation. Cependant, on a déjà conscience de certaines limites que ces outils comportent. Certains d'entre eux ont été mentionnés par les historiens des bibliothèques, du livre ou de l'édition. \newline

En premier lieu, la question des catalogues de bibliothèques est fréquemment abordée dans les divers articles traitant de leur histoire – et plus particulièrement de leurs études – tout au long du XIXe siècle. À titre d'illustration, pour trois importantes bibliothèques parisiennes telles que Sainte-Geneviève, l'Arsenal et la Mazarine :
\begin{quote}
    « La question des catalogues, déjà étudiée par le comité central des bibliothèques en 1881, puis par une commission pour les questions relatives à l’unification des catalogues des Bibliothèques publiques de Paris nommée le 27 mai 1898 et placée sous la présidence de Léopold Delisle, n’a toujours pas trouvé de solution à la veille de la Première Guerre mondiale. »,
    \end{quote}expliquent Thérèse Charmasson et  Catherine Gaziello. \newline
    
Dans la première moitié du XIXe siècle, à la Bibliothèque nationale, une controverse s'engage sur la décision à prendre : catalogue méthodique ou ordre alphabétique d'auteurs des ouvrages? On retient alors la première solution et deux catalogues méthodiques apparaissent : l'un sur l'histoire de France, l'autre sur les sciences médicales. Le résultat n'est pas satisfaisant : pour une grande partie des fonds, la référence de l'ouvrage et l'emplacement de celui-ci sont inconnus. Par la suite, il est décidé de constituer un catalogue par ordre alphabétique des auteurs : le premier volume du Catalogue général des livres imprimés est publié en 1897, le dernier est terminé en 1981. \newline

Pour pouvoir s’y retrouver et rendre l’information accessible aux acteurs la nécessitant, des systèmes de classification ont progressivement été mis en place. Dès le XIXe siècle, les catalogues imprimés sont devenus des outils pour le classement et l'accès aux collections. Ces catalogues répondaient à des normes de description, qui ont évolué au fil du temps.\newline
Du côté du monde anglosaxon, l’introduction de la \emph{Dewey Decimal Classification} en 1876 a permis de systématiser l’organisation des collections en bibliothèques, en établissant une hiérarchie de sujets pour faciliter la recherche. La classification décimale de Dewey (CDD) est un système visant à classer l’ensemble du fonds documentaire d’une bibliothèque, développé par Melvil Dewey, un bibliographe américain.
Cette classification s’appuie sur dix classes correspondant à neuf disciplines littéraires : la  philosophie, la religion, les sciences sociales, les langues, les sciences pures, les techniques, les beaux-arts et loisirs, la littérature,la géographie et l'histoire, auxquelles s’ajoute une classe « généralités ». Les subdivisions suivantes sont 10 classes, 100 divisions et 1 000 sections. \newline

En 1895, Paul Otlet et Henri La Fontaine fondent l’Institut international de bibliographie (IIB), dont les objectifs consistent à <<perfectionner et à unifier les méthodes bibliographiques et documentaires, à organiser la coopération scientifique internationale en vue d’élaborer des travaux d’ensemble, et spécialement un Répertoire bibliographique universel (RBU). Ce travail doit être établi sur fiches individuelles, et fondé sur l’emploi de la classification décimale de Dewey pour le classement méthodique de ces fiches.>>\footcite{fayet_scribe_2000_2}\newline

Comme nous avons pu le voir au travers de ces différentes évolutions et initiatives, les méthodes de gestion de l’information ont fait l’objet de siècles de maturation dans les bibliothèques. Cela participe au développement de la science de l’information. Ces répertoires représentaient une forme de mise en données avant l’heure, car ils imposent un cadre commun pour la description des objets, permettant ainsi de s’assurer qu’une information reste accessible et recherchable au sein d’une masse.\newline

\emph{La science de l’information}, qui se développe en parallèle avec les technologies et techniques de l’information, joue un rôle crucial dans la transition, du papier vers le numérique. Elle cherche à organiser et structurer l’immense masse de données accumulées par les institutions culturelles. C’est à travers des systèmes de classification, des répertoires normalisés, et des standards que la science de l’information permet de transformer ces données en un patrimoine intellectuel accessible.\footcite{fondin_si_2006}\newline

Nous pouvons voir que la réflexion autour de la normalisation de l’information pour faciliter la recherchabilité de celle-ci, est présente depuis très longtemps dans le domaine des bibliothèques. On peut donc se demander pourquoi ce n’est pas le cas dans le domaine des musées. Il faut dire que la culture du partage de l'information semble être plus présente dans les bibliothèque et les archives. Les musées semble avoir acquis cette culture plus tardivement. \newline
Mais d’abord nous allons explorer pourquoi les institutions ont un intérêt à normaliser et standardiser leurs données.

