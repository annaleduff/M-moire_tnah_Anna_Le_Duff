\textbf{Résumé en français :}
Ce mémoire explore la normalisation des données culturelles, avec un focus particulier sur l'utilisation du modèle LIDO (Lightweight Information Describing Objects) en temps que modèle de circulation de données pour l'agrégation des informations muséales par le Ministère de la Culture. Dans un contexte où les institutions culturelles font face à une fragmentation des données, le travail examine comment standardiser ces informations pour les rendre interopérables et exploitables à des fins de recherche, de diffusion ou de réutilisation. Le projet met en lumière les défis techniques liés à l'hétérogénéité des formats et des systèmes utilisés, ainsi que l'importance de l'agrégation à l'échelle nationale et européenne. \newline

\textbf{Résumé en anglais :}
This thesis explores the standardization of cultural data, with a particular focus on the use of the LIDO model (Lightweight Information Describing Objects) as a data circulation model for the aggregation of museum information by the Ministry of Culture. In a context where cultural institutions face data fragmentation, the work examines how to standardize this information to make it interoperable and usable for research, dissemination, or reuse. The project highlights the technical challenges related to the heterogeneity of formats and systems used, as well as the importance of aggregation at both national and European levels.