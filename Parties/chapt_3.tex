\subsection{L’impact des technologies numériques sur la normalisation des données}
Avec le développement des technologies numériques, la naissance d’Internet et l'apparition du Web de données,la normalisation a pris une nouvelle dimension. Les bases de données numériques ont remplacé les répertoires papier, permettant de stocker, organiser et rechercher des quantités massives de données de manière beaucoup plus efficace. Ce passage au numérique a également favorisé la formalisation de la normalisation, avec la création de standards techniques et de langages communs, tels que MARC (Machine-Readable Cataloging) pour les bibliothèques, ou EAD (Encoded Archival Description) pour les archives. \newline

Ces standards sont développés pour répondre aux besoins spécifiques des différents secteurs culturels. MARC (Machine-Readable Cataloging), par exemple, est un format utilisé principalement par les bibliothèques pour structurer les métadonnées des catalogues. Ce format  à été finalisé en 1969. Le format EAD (Encoded Archival Description), quant à lui, est un standard utilisé pour décrire les collections d'archives. Il à été développé dans les années 1990 à l’initiative de la bibliothèque de l’Université Berkeley. Nous reviendrons sur les spécificités techniques de ces formats dans notre deuxième partie.
Ces modèles permettent de structurer les données brutes métiers de manière à ce qu'elles puissent être facilement diffusées et partagées sur le Web, tout en garantissant leur interopérabilité et leur cohérence. Ils permettent de répondre à des exigences scientifiques de plus en plus hautes et doivent faciliter le travail des bibliothécaires et archivistes en leur donnant notamment, un simplifié à l'information. \newline

Les technologies numériques ont également permis l’émergence de nouveaux outils de normalisation, comme les ontologies et les thésaurus numériques. En informatique quand on parle d’ontologie on parle d’un <<corpus structuré de concepts, qui est modélisé dans un langage permettant l’exploitation par un ordinateur des relations sémantiques ou taxonomiques établies entre ces concepts.>>\footcite{ontologie} L’ontologie est construite pour un groupe de données issues d’un domaine de connaissance, ou de plusieurs domaines qui ont des liens entre eux. Le thésaurus est un <<langage documentaire servant à l'indexation de documents ou de questions afin d'alimenter et d'exploiter un système documentaire, aujourd'hui informatisé.>> \footcite{thesaurus}Ce langage prend la forme d’une liste de termes contrôlés et normalisés.
Ces outils permettent de structurer les connaissances de manière plus flexible et plus riche, en capturant les relations complexes entre les concepts et les objets.\newline

Pour permettre ces développements, des groupes de recherches se sont créés. 
\textbf{Les groupes de recherche et les standards} : vers une normalisation mondiale. \newline

Divers groupes de recherche et organisations internationales, tels que l'ICOM (International Council of Museums) et l'IFLA (International Federation of Library Associations and Institutions), ont joué un rôle crucial dans le développement de ces standards. Par exemple, le CIDOC CRM (Conceptual Reference Model), développé par l'ICOM, est un standard qui modélise les concepts et les relations dans les musées, facilitant l’échange et l’interprétation des données entre institutions. Elle est formalisée en 2006 auprès de l'Organisation internationale de normalisation (ISO) sous la référence ISO 21127.
