\subsection{Historique de la normalisation dans les musées français.}
Comparés aux bibliothèques et aux archives, les musées ont adopté les pratiques de normalisation avec un certain retard. Cependant, un certain nombre d’institutions semblent avoir pris conscience de cet enjeu, notamment avec la multiplication des initiatives de numérisation des collections.\footcite{numerisation} En France, le Service des Musées de France (SMF) a joué un rôle capital dans la promotion de la numérisation des collection et de la normalisation des données muséales. Fondé en 2009 pour succéder à la Direction des musées de France, il a imposé des normes pour l’inventaire et le récolement des collections. \footcite{art_4_2020} \footcite{loi_2002} Ce processus a permis d’intégrer les collections des musées de France dans des catalogues collectifs, facilitant ainsi leur gestion et leur valorisation à l’échelle nationale.\newline

\textbf{Le récolement : un enjeu de normalisation et de gestion des collections}\newline

Le récolement, c’est-à-dire la vérification périodique des collections, est une procédure imposée par la loi pour les musées de France. Cette procédure est un exemple concret de la normalisation appliquée à la gestion des collections. Elle permet non seulement de vérifier l’état et l’existence des objets, mais aussi de mettre à jour les informations les concernant, assurant ainsi la qualité et l’exactitude des données dans les systèmes d’information des musées. Il permet aussi parfois de se rendre compte d’anomalies, comme des détériorations d’un objet, une disparition, ou un manque d’information (pas d’acte d’entrée dans les collections par exemple).\newline

Le récolement implique de « vérifier, sur pièce et sur place, à partir d'un bien ou de son numéro d'inventaire, la présence du bien dans les collections, sa localisation, son état, son marquage, la conformité de l'inscription à l'inventaire avec le bien et, le cas échéant, avec les différentes sources documentaires, archives, dossiers d'œuvre, catalogues ». (Décret du 25 mai 2004 établissant les règles techniques concernant la gestion de l'inventaire, le registre des biens déposés dans un musée de France et le récolement).\newline

Le récolement suit des normes techniques fixées par l’Arrêté du 25 mai 2004. Ainsi, pour chaque objet récolté, on doit disposer au minimum de certaines données, comme la date d’acquisition, le numéro d’inventaire, etc…\newline

Le récolement est devenu obligatoire et systématique depuis la loi de 2002 sur les musées de France. Cette dernière a imposé de faire le récolement complet des collections tous les dix ans (on parle de récolement décennal). Il s’agit d’une mission permanente.\footcite{recolement}\newline

L'évolution de la documentation des collections muséales.\newline

\begin{quote}
    <<C’est avec la mise en place de bases de données dédiées à l’administration et à la gestion de la collection que les technologies informatiques et les politiques de numérisation entrent au musée. Traditionnellement reportés dans un cahier d’inventaire, les artefacts d’une collection sont reliés à tout élément documentaire et de gestion par un numéro, le numéro d’inventaire. La base de données informatique fonctionnant sur un principe identique (un numéro reliant des fiches contenant elles-mêmes des champs associés à un même sujet), elle devient l’outil idéal de gestion des collections. Ce premier pas désigné comme l’informatisation ou la numérisation des collections aura des conséquences sur la production des contenus associés aux artefacts ainsi que sur leur mise en forme. La base de données informatisée va permettre d’entrer dans la collection par de nouvelles entrées. À la requête par numéro d’inventaire, s’ajoutera l’emploi de thésaurus et autres systèmes de descripteurs.>> \footcite{andreacola_2014}
\end{quote}
L’histoire de la normalisation dans les musées remonte à plusieurs siècles, bien avant l’avènement du numérique. Dans les premières phases de développement des musées, la documentation des collections se faisait de manière très locale. Chaque institution avait son propre système de catalogage, qui consistait souvent en des registres, manuels ou des livres d’inventaire. Ces documents, bien que précieux, avaient des limites évidentes : ils étaient statiques, difficiles à mettre à jour, et surtout peu compatibles d’une institution à l’autre. Si un chercheur voulait consulter les informations sur une collection dans un autre musée, il devait se déplacer physiquement et se plonger dans les registres souvent non standardisés.\newline

Cette situation a perduré jusqu’à l’émergence des normes internationales en matière de gestion de l’information culturelle. Les bibliothèques et les archives ont été les premières à développer des standards pour la description des documents, avec des initiatives comme MARC pour les bibliothèques dans les années 1960. Les musées, quant à eux, ont pris du retard dans l’adoption de pratiques similaires, en raison de la complexité et de la diversité des objets qu’ils gèrent. Contrairement aux livres ou aux documents d’archives, qui sont souvent structurés de manière similaire, les objets de musée peuvent être extrêmement variés, allant des œuvres d’art aux artefacts historiques, en passant par les objets ethnographiques, scientifiques ou industriels.\newline

Cette diversité a compliqué la tâche de normalisation, car chaque type d’objet nécessite des méthodes de description spécifiques. Par exemple, décrire un tableau implique de mentionner des informations sur l’artiste, le matériau utilisé, et l’école artistique, tandis qu’un artefact archéologique peut nécessiter des informations sur son contexte de découverte, son lieu d’origine, et son utilisation historique. Cette hétérogénéité a retardé l’adoption de normes communes, chaque musée développant souvent son propre système de gestion des collections.\newline

L'impact de la normalisation internationale : le CIDOC CRM et LIDO
L’un des tournants majeurs dans l’histoire de la normalisation des données muséales a été la création du CIDOC CRM (Conceptual Reference Model) dans les années 1990 par le Conseil International des Musées (ICOM). Ce modèle conceptuel a été conçu pour répondre aux besoins des musées en matière de documentation des collections. Contrairement à des modèles plus simples comme MARC, le CIDOC CRM prend en compte la complexité des objets muséaux et des événements associés à ces objets (leur création, leur usage, leur acquisition, etc.).\newline

Le CIDOC CRM offre un cadre flexible qui permet de modéliser non seulement les objets eux-mêmes, mais aussi leur contexte historique et les événements auxquels ils sont liés. Par exemple, il peut décrire un tableau non seulement comme un objet physique, mais aussi en tant que témoin d’un événement artistique (l’acte de peindre), tout en prenant en compte les relations entre l’artiste, le commanditaire, le lieu de création, et l’œuvre elle-même. Ce modèle permet donc de capturer la richesse des collections muséales, en tenant compte des différentes couches d’information qui entourent chaque objet. \newline

Cependant, le CIDOC CRM, bien qu’efficace, est aussi un modèle complexe qui demande des compétences techniques pour être mis en œuvre et même pour être compris par les responsables de gestion interne aux musées. De nombreux musées, en particulier les plus petits, n’ont pas les ressources nécessaires pour adopter ce modèle dans sa totalité. C’est pourquoi des alternatives plus simples ont été développées pour faciliter l’adoption de normes sans pour autant sacrifier la richesse des données. Mais il faut souligner qu’en France les musées ne s’appuient pas forcément sur des modèles de données communs, mais plutôt sur les possibilités proposées par les logiciels de gestion des collections. \newline

C’est dans ce contexte qu’est né le modèle LIDO (Lightweight Information Describing Objects). Développé également par l’ICOM, LIDO est conçu pour être un modèle léger et facile à utiliser, tout en conservant la capacité de décrire des objets complexes. Contrairement au CIDOC CRM, qui est un modèle conceptuel, LIDO est un schéma XML directement exploitable dans les systèmes de gestion de collections. Il est adapté à au transport de données, à leur collecte et donc l’agrégation des données, c’est-à-dire à la collecte de données provenant de différentes sources pour les diffuser à travers des plateformes partagées, comme Europeana ou la plateforme française POP.\newline

LIDO permet aux musées de standardiser la description de leurs collections de manière simple et efficace. Par exemple, un musée peut utiliser LIDO pour exporter ses données vers une plateforme nationale ou internationale, garantissant ainsi que les informations partagées sont conformes aux demandes de celles-ci. Le LIDO peut aussi être utilisé pour faire le lien entre différentes institutions entre elles, lors de prêts pour des expositions par exemple. LIDO prend en charge des éléments essentiels de la description d’objets, tels que le titre, l’artiste, la date de création, le matériau, les dimensions, ainsi que des informations sur les droits et l’accès aux œuvres. En ce sens, LIDO est un outil puissant pour répondre aux besoins d’interopérabilité, tout en étant plus accessible que des modèles plus complexes comme le CIDOC CRM. \newline

\textbf{Les défis de l’adoption des standards de données culturelles en France.}\newline

En France, la normalisation des données muséales a été portée par des initiatives gouvernementales, notamment sous l’impulsion du Ministère de la Culture. Le Service des Musées de France (SMF) a joué un rôle clé dans la promotion de la standardisation à l’échelle nationale. Le SMF est responsable de l’encadrement de la gestion des musées de France, un réseau qui regroupe environ 1 200 musées sur l’ensemble du territoire français.\newline

L’un des premiers efforts de normalisation en France a été la création de la base de données Joconde, qui regroupe les collections des musées français sous un format standardisé. Joconde a été conçue comme une plateforme nationale permettant de centraliser les informations sur les collections des musées, de les rendre accessibles au grand public et aux chercheurs, et de faciliter la collaboration entre les institutions. Cette initiative a marqué un tournant dans la manière dont les musées français gèrent leurs collections, en introduisant un format d’export standardisé pour la description des objets. \newline

Aujourd’hui pour être intégrée à la base Joconde et diffusée sur POP, une notice d’objet doit respecter une certaine forme. Cette forme implique une rigueur scientifique puisque la notice doit comporter un minimum d’information. Parmis ces données obligatoires on compte le numéro d’inventaire, le domaine (vocabulaire contrôlé), et le statut juridique. D’autres données ne sont pas signalées comme obligatoires mais comme indispensables, c’est le cas pour la désignation, les mesures ou le lieu de conservation, entre autres.\newline

Cependant, malgré ces avancées, la diversité des pratiques reste un défi majeur pour l’adoption des standards en France. Chaque musée, qu’il soit national, régional ou local, a ses propres méthodes de gestion des collections, souvent en fonction des ressources disponibles et de la nature de ses collections. Certains musées, notamment les plus petits, n’ont pas toujours les moyens de mettre en place des systèmes de gestion sophistiqués, et continuent d’utiliser des méthodes plus traditionnelles ou des solutions informatiques non standardisées.\newline

De plus, la diversité des fournisseurs de logiciels pour la gestion des collections complique l’adoption de standards communs. De nombreux musées utilisent des logiciels propriétaires développés par des entreprises spécialisées, qui ne sont pas toujours compatibles avec les normes nationales ou internationales. Cette situation conduit à une diversité de mode de description des œuvres et donc à fragmentation des données, chaque musée ayant son propre système de gestion, ses propres structurations de leur base de données des collections et de leur mode de description, et ses propres pratiques de documentation. Cela rend difficile l’intégration de ces données dans des plateformes partagées comme POP, qui exige des formats standardisés. Qui plus est, l'obligation de récolement décennal rend le processus de partage de données vers POP, très lourd administrativement et le versement des données est souvent ralenti car les données ne sont pas à jour si récolement pas terminé. Des échanges avec des professionels nous on permis d'estimer que pour certaines institutions muséales le retard potentiel serait de 10 ans ou plus sur les informations disponible sur le portail.\newline

Pour répondre à ces défis, le Ministère de la Culture a mis en place des recommandations pour encourager les musées à adopter des standards communs, en particulier LIDO. Des formations sont également proposées pour aider les musées à s’adapter aux nouvelles exigences de l’interopérabilité et de la gestion des données numériques. L’objectif est de créer un écosystème unifié, où toutes les institutions peuvent partager leurs données de manière fluide et efficace, tout en préservant la spécificité de leurs collections.
